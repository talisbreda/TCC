\setlength{\parskip}{0.2cm}
\section*{\centering RESUMO}
\noindent Problemas de emparelhamento em grafos, definidos como uma seleção de conjuntos de arestas sem vértices em comum, possuem ampla relevância em áreas da computação, como visão computacional, e também fora dela, como na biologia. São bastante importantes em tarefas que envolvem correspondência de elementos, como a comparação e correspondência de elementos em conjuntos distintos, por exemplo em análise de dados estruturados. Enquanto as formulações clássicas são solúveis eficientemente, variações com restrições adicionais pertencem frequentemente à classe NP-Difícil, exigindo métodos inovadores para equilibrar precisão e eficiência. Considerando a quantidade de algoritmos e métodos presentes na literatura, torna-se difícil encontrar a maneira mais eficiente de resolver um problema específico. Este trabalho busca levantar, comparar e classificar métodos computacionais, destacando suas aplicações e limitações, com o objetivo de fornecer insights úteis para pesquisadores e profissionais na escolha de soluções eficazes para cenários práticos.\par

\noindent\textbf{Palavras-chave:} Emparelhamento em grafos, grafos bipartidos, visão computacional
\newpage

\section*{\centering ABSTRACT}
\noindent Graph matching problems, defined as a selection of edges sharing no common vertices, hold significant relevance in computational fields such as computer vision, as well as in other domains like biology. They are crucial in tasks involving element correspondence, such as the comparison and matching of elements across distinct sets—for instance, in the analysis of structured data. While classical formulations are efficiently solvable, variations with additional constraints often belong to the NP-Hard class, necessitating innovative methods to balance accuracy and efficiency. Given the vast number of algorithms and methods available in the literature, identifying the most efficient approach for a specific problem becomes challenging. This work aims to survey, compare, and classify computational methods, highlighting their applications and limitations, with the goal of providing valuable insights to assist researchers and practitioners in selecting effective solutions for practical scenarios. \par

\noindent\textbf{Keywords:} Graph matching, bipartite graphs, computer vision
\newpage
\setlength{\parskip}{0pt}
