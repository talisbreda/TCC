\setlength{\parskip}{0.2cm}
\section*{\centering RESUMO}
\noindent Problemas de emparelhamento em grafos, definidos como uma seleção de conjuntos de arestas sem vértices em comum, possuem ampla relevância em áreas da computação, como visão computacional, e também fora dela, como na biologia. São bastante importantes em tarefas que envolvem correspondência de elementos, como a comparação e correspondência de elementos em conjuntos distintos, por exemplo em análise de dados estruturados. Devido à complexidade computacional frequentemente NP-completa desses problemas, são necessárias abordagens que conciliem precisão e eficiência. Considerando a quantidade de algoritmos e métodos presentes na literatura, torna-se difícil encontrar a maneira mais eficiente de resolver um problema específico. Este trabalho busca levantar, comparar e classificar métodos computacionais, destacando suas aplicações e limitações, com o objetivo de fornecer insights úteis para pesquisadores e profissionais na escolha de soluções eficazes para cenários práticos.\par

\noindent\textbf{Palavras-chave:} Emparelhamento em grafos, grafos bipartidos, visão computacional
\newpage

\section*{\centering ABSTRACT}
\noindent Graph matching problems, defined as the selection of edge sets with no common vertices, are highly relevant in fields of computer science such as computer vision, as well as in areas outside of it, like biology. They are particularly important for tasks involving element matching, such as comparing and matching elements in distinct sets, for example, in structured data analysis. Due to the often NP-complete computational complexity of these problems, approaches that balance precision and efficiency are necessary. Given the large number of algorithms and methods in the literature, it is challenging to determine the most efficient way to solve a specific problem. This work aims to survey, compare, and classify computational methods, highlighting their applications and limitations, with the goal of providing useful insights for researchers and professionals in selecting effective solutions for practical scenarios.\par

\noindent\textbf{Keywords:} Graph matching, bipartite graphs, computer vision
\newpage
\setlength{\parskip}{0pt}
