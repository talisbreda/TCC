%!TeX root = ../main.tex


% ==================================================================================================================================
% 3. Introdução sobre taxonomia dos problemas
% ==================================================================================================================================
\section{Taxonomia dos problemas}

No contexto de emparelhamento em grafos, é possível classificar os problemas em diferentes categorias com base nas características dos grafos e nos objetivos específicos de emparelhamento. 

\subsection{Classificação Baseada na Estrutura do Grafo e Tipo de Atribuição}

A classificação inicial dos problemas de emparelhamento é frequentemente determinada pela estrutura do grafo subjacente e pelas 
restrições de atribuição impostas:

\begin{itemize}
    \item \textbf{Grafos Bipartidos: } É um caso restrito, mas crucial no estudo do problema de emparelhamento. Pode ser resolvido 
    de forma exata, em um "ambiente" contido, utilizando algoritmos como o algoritmo Hungaro e algoritmos de fluxo em redes.
    \cite{nips2006} \cite{lawler}
    \item \textbf{Grafos Gerais: } O emparelhamento em grafos não bipartidos é uma generalização mais complexa do problema. Encontrar uma solução exata para este tipo de problema pode ser computacionalmente desafiador, e muitas vezes são utilizados algoritmos aproximados ou heurísticas para obter soluções de forma viável.
    \item \textbf{Multigrafos: } Alguns problemas de emparelhamento envolvem multigrafos, onde múltiplas arestas podem existir entre 
    o mesmo par de vértices. Como mencionado anteriormente, este trabalho não abordará esse tipo de problema.
\end{itemize}