\section{Problemas de Emparelhamento}

% ==================================================================================================================================
% 3.1. Introdução sobre taxonomia dos problemas
% ==================================================================================================================================
\subsection{Introdução sobre taxonomia dos problemas}

% ==================================================================================================================================
% 3.2. Emparelhamento em grafos bipartidos
% ==================================================================================================================================
\subsection{Emparelhamento em grafos bipartidos}

\subsubsection{Emparelhamento bipartido}

% ---------------------------------------------------------------------------------------------------------------------------------
% 3.2.1. Emparelhamento de cardinalidade máxima
% ---------------------------------------------------------------------------------------------------------------------------------
\subsubsection{Emparelhamento de cardinalidade máxima}

Dado um grafo bipartido, o problema da cardinalidade máxima busca encontrar o emparelhamento que contém o maior número possível de 
arestas. Ele pode ser reduzido ao problema de fluxo máximo em uma rede da seguinte maneira:

Construimos um grafo direcionado a partir do grafo bipartido original, adicionando um vértice fonte $s$ e um vértice sumidouro $t$.
Conectamos o vértice fonte $s$ a todos os vértices do primeiro conjunto de partição com arestas de capacidade 1, e conectamos
todos os vértices do segundo conjunto de partição ao vértice sumidouro $t$ com arestas de capacidade 1. As arestas entre os dois
conjuntos de partição recebem capacidade infinita. Seguindo o teorema da integralidade \textbf{CITAR FONTES}, qualquer fluxo máximo encontrado nesta rede
será integral, correspondendo a um emparelhamento no grafo bipartido original.

Aplicando um algoritmo de fluxo máximo, como o algoritmo de Edmonds-Karp ou o algoritmo de Dinic, podemos determinar o fluxo máximo
da fonte ao sumidouro. O conjunto de arestas que transportam fluxo na solução do fluxo máximo forma o emparelhamento de 
cardinalidade máxima no grafo bipartido original. \textbf{CITAR FONTES}

% ---------------------------------------------------------------------------------------------------------------------------------
% 3.2.1. Problema de atribuição (Assignment Problem)
% ---------------------------------------------------------------------------------------------------------------------------------
\subsubsection{Problema de atribuição (Assignment Problem)}
\textbf{Talvez chamar de Problema de emparelhamento ponderado?}

O problema de atribuição consiste em encontrar uma combinação ótima de atribuições entre dois conjuntos disjuntos, minimizando 
o custo total associado a essas atribuições. Exemplo: Considere $N$ trabalhadores e $N$ tarefas, onde cada trabalhador pode ser designado 
a exatamente uma tarefa, e cada tarefa deve ser atribuída a exatamente um trabalhador. O custo de atribuir o trabalhador i à tarefa j é 
representado por uma matriz de custos $C = [c_{ij}]$ \cite{lawler}. O objetivo é encontrar um conjunto de atribuições que minimize o custo total.

É possível modelar esse problema como um problema de emparelhamento de custo mínimo em um grafo bipartido ponderado, da seguinte forma:

Construimos um grafo bipartido com $N$ vértices em cada conjunto de partição, onde cada vértice do 
primeiro conjunto representa um homem e cada vértice do segundo conjunto representa um trabalho. Cada aresta entre um vértice do 
primeiro conjunto e um vértice do segundo conjunto é ponderada com o custo $c_{ij}$ de atribuir o homem i ao trabalho j. 
Cria-se um novo vértice fonte conectado a todos os vértices do primeiro conjunto com arestas de peso zero, e um novo vértice 
sumidouro conectado a todos os vértices do segundo conjunto com arestas de peso zero. O objetivo é encontrar um fluxo máximo de 
custo mínimo neste grafo modificado, que corresponde a um emparelhamento de custo mínimo no grafo bipartido original \cite{lawler}.
Assim, reduzimos o problema de emparelhamento em um grafo bipartido ponderado ao problema de fluxo de custo mínimo, que pode ser 
resolvido eficientemente usando algoritmos de fluxo de custo mínimo, como o algoritmo de ciclo negativo \textbf{CITAR FONTES}

% ---------------------------------------------------------------------------------------------------------------------------------
% 3.2.1. Bottleneck assignment problem
% ---------------------------------------------------------------------------------------------------------------------------------
\subsubsection{Problema de atribuição gargalo}

O problema de atribuição gargalo, também conhecido como \textbf{Problema de Emparelhamento Min-Max} ou \textbf{Bottleneck Assignment
problem}, é uma variação do problema de atribuição tradicional. O objetivo deste problema é encontrar, em um grafo bipartido ponderado,
um emparelhamento de cardinalidade máxima (maior número possível de arestas) no qual o mínimo dos pesos das arestas do emparelhamento seja
maximizado \cite{lawler}.

Por exemplo. Considere $N$ trabalhadores e $N$ estações de trabalho. $w_{ij}$ representa a eficiência do trabalhador i na 
estação de trabalho j. A eficiência total da produção é limitada pela eficiência do trabalhador menos eficiente. 
O objetivo é atribuir trabalhadores às estações de trabalho de forma a maximizar a eficiência mínima entre todas as atribuições. 
Para resolver este problema, podemos utilizar o método de threshold \cite{lawler}.

\subsubsection{Problema do emparelhamento estável (Stable Marriage Problem)}

\subsubsection{Problema de atribuição quadrática (QAP)}

\subsubsection{Problema de atribuição linear (LAP)}