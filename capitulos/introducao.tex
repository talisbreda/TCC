\section{Introdução}

Um emparelhamento em um grafo não-dirigido é definido como um conjunto de arestas sem pontas em comum. Em outras palavras, um emparelhamento é um conjunto $M$ 
de arestas que satisfaz a seguinte propriedade: o grau de cada vértice no subconjunto $M$ é no máximo 1 \cite{imeUSP_matching}. Os principais problemas relacionados 
a emparelhamentos geralmente envolvem grafos bipartidos, aqueles que podem ser particionados em dois conjuntos independentes $U$ e $V$, onde todas as arestas 
conectam vértices de $U$ com vértices de $V$.

Os problemas de emparelhamento têm aplicações significativas em várias áreas, como biologia computacional, onde podem ser empregados na diferenciação e 
classificação de estruturas proteicas \cite{Taylor2002}; redes sociais, para identificação de comunidades ou grupos \cite{wenfei_socialnetwork}; e visão 
computacional, para correspondências entre elementos de conjuntos distintos, como pontos em imagens ou vértices em malhas tridimensionais 
\cite{haller2022comparativestudygraphmatching}.

Resolver problemas de emparelhamento em grafos é um desafio devido à sua complexidade computacional, frequentemente NP-completa, exigindo métodos inovadores 
para equilibrar precisão e eficiência. Estratégias clássicas, como o Problema de Atribuição Quadrática (QAP), modelam o emparelhamento como uma questão de 
otimização combinatória, empregando relaxações espectrais e heurísticas para encontrar soluções aproximadas de forma eficiente \cite{ijcai2020p694}. 
Por outro lado, avanços em transporte ótimo duplamente estocástico, como o algoritmo GOAT, têm demonstrado melhorias em robustez e velocidade, especialmente 
em grafos maiores \cite{saadeldin_goat}.

Estudos como esses ilustram a importância de investigar e comparar métodos computacionais que abordem a complexidade inerente ao problema de emparelhamento em 
grafos, contribuindo para aplicações como aprendizado de máquina e visão computacional. Essa abordagem torna-se essencial para desenvolver soluções adaptadas 
a diferentes cenários e com impacto direto na melhoria de algoritmos existentes.

Diante desse cenário, este trabalho visa sistematizar os principais problemas e métodos computacionais associados ao emparelhamento em grafos. 
Ele busca oferecer uma comparação detalhada entre suas aplicações e limitações, contribuindo para o entendimento aprofundado das ferramentas disponíveis 
e identificando oportunidades para o desenvolvimento de novas soluções. Essa abordagem torna-se ainda mais relevante à luz da importância prática desses 
algoritmos para resolver problemas reais em grande escala, conforme evidenciado por estudos recentes \cite{MIT2021}.

\subsection{Objetivos}

\subsubsection{Objetivo geral}

Este trabalho tem como objetivo principal levantar, organizar e comparar problemas de emparelhamento em grafos para diversos casos de uso, além de comparar 
diferentes métodos computacionais presentes na literatura para a solução desses problemas.

\subsubsection{Objetivos específicos}

\begin{itemize}
    \item Mapear as variações e características de diferentes problemas de emparelhamento presentes na literatura.
    \item Investigar e classificar métodos computacionais presentes na literatura de acordo com suas vantagens, limitações e casos de uso.
    \item Criar um ambiente de testes unificado a fim de reduzir variabilidade nos resultados e proporcionar comparações consistentes
    \item Implementar e testar os algoritmos descritos, a fim de comparar desempenho e eficiência em diferentes situações
    \item Organizar os resultados em um formato que facilite o acesso e a compreensão dos tópicos abordados por pesquisadores e profissionais.
\end{itemize}

\subsection{Delimitação do estudo}

Os problemas de emparelhamento em grafos podem ser classificados de diversas maneiras, como sendo em grafos bipartidos ou não-bipartidos, ponderados ou não-ponderados, 
direcionados ou não-dire\-cionados, problemas em multigrafos, problemas em hipergrafos, entre outros. Este trabalho focará em problemas que envolvem grafos simples, 
isto é, excluindo multigrafos e hipergrafos.

