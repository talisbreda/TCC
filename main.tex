\documentclass{article}
\usepackage{graphicx}
\usepackage{helvet}
\usepackage{lipsum}
\usepackage{float}
\usepackage{tabularx}
\usepackage{multirow}
\usepackage{multicol}
\usepackage{amssymb}
\usepackage{pifont}
\usepackage{setspace}
\usepackage{tocloft}
\usepackage{indentfirst}
\usepackage[hidelinks]{hyperref}
\usepackage[alf]{abntex2cite}
\usepackage[portuguese]{babel}
\usepackage[letterspace=100]{microtype}

% ABNT 14724: As folhas devem apresentar margem esquerda e superior de 3 cm; direita e inferior de 2 cm.
\usepackage[left=3cm,top=3cm,bottom=2cm,right=2cm]{geometry}

% Define família de fonte padrão como Helvetica
\renewcommand{\familydefault}{\sfdefault}

% Define um X para ser usado no cronograma
\newcommand\xmark{\huge{$\times$}}

% Adiciona linha pontilhada para \section no sumário (\tableofcontents)
\renewcommand{\cftsecleader}{\cftdotfill{\cftdotsep}} % Add dots for sections

\newcommand{\topic}[1]{%
  \bigskip
  \textbf{#1}
  \medskip
}

% Include numbering for paragraphs (up to 4 levels)
\usepackage{titlesec}

\setcounter{secnumdepth}{4}
\setcounter{tocdepth}{4}

% Change Section
\titleformat{\section}
  {\LARGE\bfseries} % Size: 16pt, Line-height: 18pt, Bold
  {\thesection} % The number (e.g., "1")
  {1em}         % Space between number and title
  {}  
% Change Subsection
\titleformat{\subsection}
  {\Large\bfseries} % Format: Size and Bold
  {\thesubsection}  % The number (e.g., 2.1)
  {1em}             % Space between number and title
  {}                % Code before the title body

% Change Subsubsection
\titleformat{\subsubsection}
  {\large\bfseries}
  {\thesubsubsection}
  {1em}
  {}

\titleformat{\paragraph}
{\normalfont\normalsize\bfseries}{\theparagraph}{1em}{}
\titlespacing*{\paragraph}
{0pt}{3.25ex plus 1ex minus .2ex}{1.5ex plus .2ex}

% End paragraph numbering configuration


% Listings configuration (for code blocks)

\usepackage{listings}
\usepackage{xcolor} % for colors

\lstset{
backgroundcolor=\color{lightgray},   % set background color
  basicstyle=\ttfamily\footnotesize,    % font type and size
  numbers=left,                         % line numbers on the left
  numberstyle=\tiny\color{gray},         % line number style
  keywordstyle=\color{blue},             % keyword style
  commentstyle=\color{green!50!black},   % comment style
  stringstyle=\color{red},               % string literal style
  showstringspaces=false,                % don't mark spaces in strings
  frame=single,                          % frame around the code
  breaklines=true                        % automatic line breaking
}
% End listings configuration

% algorithm2e settings

\usepackage[utf8]{inputenc}
\usepackage{amsmath}
\usepackage{amssymb} % For math fonts
\usepackage[linesnumbered,ruled,vlined]{algorithm2e}
% 1. Change line numbering style to (1), (2)...
\SetNlSty{textnormal}{(}{)}

% 2. Define the "Procedure" keyword
\SetKwProg{Fn}{Procedure}{}{}

% 3. Define specific keywords to match the image
\SetKw{KwOr}{or}
\SetKw{KwAnd}{and}
\SetKw{KwTo}{to}
\SetKw{KwStep}{step}
\SetKw{KwLet}{let}
\SetKw{KwStop}{stop}
\SetKw{KwIff}{iff}
\SetKw{KwTrue}{true}
\SetKw{KwBreak}{break}
\SetKw{KwRet}{return}

\newcommand{\AlgComment}[1]{\textbf{[comment: #1]}}

\begin{document}

% Remove contagem de páginas
\pagestyle{empty}

\newcommand{\nome}{Tális Breda}
\newcommand{\titulo}{Estudo da Família de Emparelhamento em Grafos}
\newcommand{\orientador}{Prof. Dr. Rafael de Santiago}
\newcommand{\ano}{2024}
\newcommand{\local}{Florianópolis}
\newcommand{\instituicaosigla}{UFSC}
\newcommand{\instituicao}{Universidade Federal de Santa Catarina}
\newcommand{\tipotrabalho}{Trabalho de Conclusão do Curso}
\newcommand{\formacao}{Bacharel em Ciências da Computação}
\newcommand{\curso}{Ciências da Computação}
\newcommand{\programa}{Graduação em Ciências da Computação}
\newcommand{\centro}{Centro Tecnológico}
\newcommand{\departamento}{Departamento de Informática e Estatística}
\newcommand{\sigladepartamento}{INE/UFSC}

\begin{center}
    \large
    \includegraphics[width=1.84cm, keepaspectratio]{images/ufsc.jpg}\par
    \MakeUppercase{
        \instituicao\par
        \centro\par
        \formacao\par
    }
    
    \vspace*{5\baselineskip}

    \nome

    \vspace*{8\baselineskip}

    \textbf{\MakeUppercase \titulo}\par

    \vfill

    \local\par
    \ano\par

    \newpage

    
\end{center}
\begin{center}
    \large 
    \nome\par
    \vfill
    \textbf{\MakeUppercase \titulo}\par
    \vfill
    \begin{flushright}
        \parbox{0.6\linewidth}{\small\tipotrabalho~de~\programa~do~\centro~da~\instituicao~para~a~obtenção~do~título~de~\formacao.\par
        Orientador: \orientador}
    \end{flushright}
    \vfill
    \local\par
    \ano\par
    \newpage
\end{center}
\setlength{\parskip}{0.2cm}
\section*{\centering RESUMO}
\noindent Problemas de emparelhamento em grafos, definidos como uma seleção de conjuntos de arestas sem vértices em comum, possuem ampla relevância em áreas da computação, como visão computacional, e também fora dela, como na biologia. São bastante importantes em tarefas que envolvem correspondência de elementos, como a comparação e correspondência de elementos em conjuntos distintos, por exemplo em análise de dados estruturados. Enquanto as formulações clássicas são solúveis eficientemente, variações com restrições adicionais pertencem frequentemente à classe NP-Difícil, exigindo métodos inovadores para equilibrar precisão e eficiência. Considerando a quantidade de algoritmos e métodos presentes na literatura, torna-se difícil encontrar a maneira mais eficiente de resolver um problema específico. Este trabalho busca levantar, comparar e classificar métodos computacionais, destacando suas aplicações e limitações, com o objetivo de fornecer insights úteis para pesquisadores e profissionais na escolha de soluções eficazes para cenários práticos.\par

\noindent\textbf{Palavras-chave:} Emparelhamento em grafos, grafos bipartidos, visão computacional
\newpage

\section*{\centering ABSTRACT}
\noindent Graph matching problems, defined as a selection of edges sharing no common vertices, hold significant relevance in computational fields such as computer vision, as well as in other domains like biology. They are crucial in tasks involving element correspondence, such as the comparison and matching of elements across distinct sets—for instance, in the analysis of structured data. While classical formulations are efficiently solvable, variations with additional constraints often belong to the NP-Hard class, necessitating innovative methods to balance accuracy and efficiency. Given the vast number of algorithms and methods available in the literature, identifying the most efficient approach for a specific problem becomes challenging. This work aims to survey, compare, and classify computational methods, highlighting their applications and limitations, with the goal of providing valuable insights to assist researchers and practitioners in selecting effective solutions for practical scenarios. \par

\noindent\textbf{Keywords:} Graph matching, bipartite graphs, computer vision
\newpage
\setlength{\parskip}{0pt}


\tableofcontents
\newpage

\input{capitulos/1_introducao}
%!TeX root = ../main.tex
\section{Fundamentação teórica}

% ==================================================================================================================================
% 2.1. Definições sobre grafos
% ==================================================================================================================================
\subsection{Definições sobre grafos}


Um \textbf{grafo} $G$ é uma uma estrutura definida por um par $G=(V,E)$, consistindo de um conjunto finito $V$ de elementos 
chamados \textbf{vértices} (ou \textbf{nós} ou \textbf{pontos}) e um conjunto (ou família) $E$ de pares não ordenados de vértices, 
chamados de \textbf{arestas} \cite{jungnickel}. O conjunto V é o conjunto de vértices de $G (VG)$ e $E$ é a família de arestas de 
$G (EG)$.

No contexto deste projeto, assumimos que as arestas $e={u,v}$ são pares não ordenados de vértices distintos. Se uma aresta $e$ 
conecta os vértices $a$ e $b$, diz-se que $a$ e $b$ são \textbf{adjacentes} ou \textbf{vizinhos}, e que a aresta $e$ é 
\textbf{incidente} a $a$ e $b$ \cite{schrijver}.

O conceito de grafo é amplamente utilizado como uma representação abstrata concisa para estruturas complexas e para 
codificar relações binárias entre um conjunto de objetos.

% ---------------------------------------------------------------------------------------------------------------------------------
% 2.1.1. Grafo direcionado
% ---------------------------------------------------------------------------------------------------------------------------------
\subsubsection{Grafo direcionado}

Se as relações entre os vértices forem assimétricas, utilizamos um \textbf{grafo direcionado} ou \textbf{dígrafo}, $D=(V,A)$. 
Neste caso, $A$ é uma família de pares ordenados de vértices, chamados \textbf{arcos} (ou arestas direcionadas). 
Para um arco $e=(u,v)$, $u$ é chamado o vértice inicial ou \textbf{cauda} (tail), e $v$ é o vértice final ou \textbf{cabeça} (head)
\cite{manber}

% ---------------------------------------------------------------------------------------------------------------------------------
% 2.1.2. Grafo ponderado
% ---------------------------------------------------------------------------------------------------------------------------------
\subsubsection{Grafo ponderado}

Um grafo $G=(V,E)$ é chamado \textbf{grafo ponderado} (ou com pesos) se uma \textbf{função de peso} (ou função de custo, 
ou função de comprimento) $w:E\rightarrow R$ está associada às arestas. Formalmente: $G=(V,E,w)$, onde $w:E\rightarrow R$. 
Geralmente, o peso $w(e)$ de uma aresta $e$ é um valor real não negativo que representa o custo, comprimento ou capacidade 
associada àquela aresta \cite{rosen}.

Grafos ponderados podem ser representados por matrizes de adjacência de valor real $A$, onde a entrada $A_{ij}$ 
é o peso $w_{ij}$ da aresta \cite{path_following_algo}

% ---------------------------------------------------------------------------------------------------------------------------------
% 2.1.3. Grafo bipartido
% ---------------------------------------------------------------------------------------------------------------------------------
\subsubsection{Grafo bipartido}

Um grafo bipartido é um grafo não direcionado que pode ser facilmente colorido com apenas duas cores \cite{manber}

Formalmente, um grafo $G=(V,E)$ é chamado \textbf{grafo bipartido} se o conjunto de vértices $V$ puder ser
particionado em dois subconjuntos disjuntos $V_1$ e $V_2$, chamados calsses de cor, tais que:

\begin{equation}
    V = V_1 \cup V_2 \quad {e} \quad V_1 \cap V_2 = \emptyset
\end{equation}

e todas as arestas $e \in E$ conectam um vértice em $V_1$ a um vértice em $V_2$. Ou seja, não existem arestas que conectem
vértices dentro do mesmo subconjunto \cite{dasgupta}.

Observações:
\begin{itemize}
    \item Um grafo é bipartido se e somente se não contém ciclos de comprimento ímpar \cite{schrijver}
    \item O grafo bipartido completo $K_{m,n}$ possui $m$ vértices em $V_1$ e $n$ vértices em $V_2$, 
    e contém todas as arestas possíveis entre $V_1$ e $V_2$ \cite{jungnickel}
\end{itemize}


% TODO: teorema da integralidade
% TODO: Mendelsohn-Dulmage

\subsection{Busca em grafos}

A exploração sistemática dos vértices e arestas de um grafo é uma sub-rotina fundamental para a maioria dos algoritmos em redes, incluindo aqueles utilizados para encontrar emparelhamentos, fluxo máximo e componentes conexos. A busca em grafos consiste em seguir as arestas do grafo para visitar os vértices, permitindo descobrir a estrutura da rede e propriedades de conectividade \cite{cormen}.

\subsubsection{Representação computacional}

Antes de discutir os algoritmos de busca, é essencial definir como o grafo $G=(V,E)$ é representado computacionalmente, pois isso impacta diretamente a complexidade temporal das operações. Destacam-se duas representações primárias \cite{aho}:

\begin{itemize}
    \item \textbf{Matriz de adjacência:} Uma matriz $A$ de dimensão $|V| \times |V|$, onde $A_{ij} = 1$ se existe uma aresta $(i,j) \in E$, e $0$ caso contrário. Embora permita verificação de existência de arestas em tempo constante $O(1)$, ela consome espaço $\theta(V^2)$ ,sendo ineficiente para grafos esparsos.
    \item \textbf{Matriz de adjacência:} Consiste em um array de $|V|$ e listas, onde caad lista $Adj[u]$ contém todos os vértices $v$ tal que $(u,v) \in E$. Esta representação é preferível para grafos esparsos, pois consome espaço $\theta(V+E)$ \cite{cormen}.
\end{itemize}

Para os algoritmos de busca descritos a seguir, assume-se geralmente o uso de listas de adjacência, resultando em complexidades lineares em relação ao tamanho do grafo.

\subsubsection{Busca em largura}

A \textbf{busca em largura}, ou Breadth-First Search (BFS) é um dos algoritmos mais simples e fundamentais para a exploração de grafos. A BFS explora o grafo em "camadas" ou níveis. Dado um vértice fonte $s$, o algoritmo visita primeiro todos os vizinhos de $s$ (camada $L_1$), depois os vizinhos desses vizinhos (camada $L_2$), e assim sucessivamente \cite{kleinberg}.

O procedimento utiliza uma estrutura de dados do tipo Fila (FIFO) para gerenciar a fronteira de exploração. \citeonline{cormen} descrevem o uso de um sistema de "cores" para monitorar o estado dos vértices: brancos (não descobertos), cinzas (descobertos, mas com vizinhos ainda não explorados) e pretos (totalmente explorados).

\textbf{Propriedades e Complexidade:} A propriedade mais importante da BFS, destacada por \citeonline{dasgupta}, é que ela calcula o caminho mais curto (em número de arestas) de $s$ a todos os outros vértices alcançáveis em grafos não ponderados. A complexidade de tempo da BFS é $O(V+E)$, pois cada vértice é enfileirado no máximo uma vez e cada lista de adjacência é varrida uma única vez.

\subsubsection{Busca em profundidade}

A \textbf{busca em profundidade} ou Depth-First Search (DFS), ao contrário da estratégia por níveis da BFS, a busca em profundidade explora as arestas a partir do vértice mais recentemente descoberto que ainda possui arestas inexploradas. Conforme \citeonline{manber}, a DFS "aprofunda-se" no grafo tanto quanto possível e, quando não há mais para onde ir, realiza o backtracking (retrocesso) para explorar outros caminhos.

Computacionalmente, a DFS pode ser implementada de forma recursiva ou utilizando uma Pilha (LIFO). Uma característica crucial da DFS é a classificação temporal dos eventos. \citeonline{cormen} sugerem registrar dois carimbos de tempo para cada vértice $u$:

\begin{itemize}
    \item $d[u]$: Momento de descoberta (quando o vértice passa de branco para cinza).
    \item $f[u]$: Momento de finalização (quando o vértice passa de cinza para preto, após toda sua lista de adjacência ser explorada).
\end{itemize}

\textbf{Classificação de arestas}: a execução da DFS induz uma estrutura de floresta (Floresta DFS) e permite classificar as arestas do grafo original em quatro tipos, fundamentais para entender ciclos e estrutura \cite{cormen}
\begin{itemize}
    \item \textbf{Arestas de Árvore:} Arestas $(u,v)$ percorridas pela busca quando $v$ é descoberto.
    \item \textbf{Arestas de Retorno (Back edges):} Conectam um vértice $u$ a um ancestral $v$ na árvore DFS. A existência destas arestas indica a presença de ciclos.
    \item \textbf{Arestas de Avanço (Forward edges):} Conectam $u$ a um descendente $v$ que não é seu filho direto.
    \item \textbf{Arestas de Cruzamento (Cross edges):} Conectam vértices sem relação de ancestralidade.
\end{itemize}

A complexidade da DFS, assim como a BFS, é $\theta(V+E)$ para grafos representados por listas de adjacência.


\subsection{Caminhos mais curtos}

O problema de encontrar o caminho mais curto entre dois vértices em um grafo é um dos problemas de otimização combinatória mais estudados, servindo como sub-rotina para diversas aplicações, incluindo roteamento de redes e algoritmos de fluxo.

Seja $G=(V,E)$ um grafo ponderado com uma função de peso $w : E \rightarrow \mathbb{R}$. O peso de um caminho $p = \langle v_0, v_1 \dots, v_k \rangle$ é a soma dos pesos de suas arestas constituintes: $w(p) = \sum_{i=1}^{k}w(v_{i-1},v_i)$. O \textbf{caminho mais curto} de um vértice $u$ a um vértice $v$ é definido como qualquer camino $p$ com peso mínimo $\delta(u,v)$ \cite{cormen}.

\subsubsection{Princípio do relaxamento e subestrutura ótima}

Os algoritmos de caminho mínimo baseiam-se na propriedade de \textbf{subestrutura ótima}: subcaminhos de um caminho mais curto são, por si mesmos, caminhos mais curtos. Além disso, utilizam a técnica de \textbf{relaxamento} \cite{kleinberg}.

Para cada vértice $v$, o algoritmo mantém um atributo $d[v]$, que é um limite superior para o peso do caminho mais curto da fonte $s$ até $v$. O processo de relaxar uma aresta $(u,v)$ consiste em testar se é possível melhorar o caminho para $v$ passando por $u$:

\begin{equation}
    \text{Se } d[v] > d[u] + w(u,v)\text{, então } d[v] \leftarrow d[u] + w(u,v)
\end{equation}

\begin{algorithm}[H]
    \DontPrintSemicolon
    \SetAlgoLined
        \Fn{Relax($u, v, w$)}{
        \If{$v.d > u.d + w(u, v)$}{
            $v.d \leftarrow u.d + w(u, v)$\;
            $v.\pi \leftarrow u$\;
        }
    }
    \caption{Método de relaxamento \cite{cormen}}
\end{algorithm}

A diferença entre os algoritmos reside na ordem e na frequência com que as arestas são relaxadas.

\subsubsection{Algoritmo de Dijkstra}

Para grafos onde os pesos das arestas são não-negativos ($w(u,v)\geq 0$), o Algoritmo de Dijkstra é a abordagem mais eficiente. Ele utiliza uma estratégia gulosa, mantendo um conjunto de vértices cujos pesos finais dos caminhos mais curtos já foram determinados \cite{aho}.

O algoritmo seleciona repetidamente o vértice $u$ com a menor estimativa de caminho mais curto $d[u]$ de uma fila de prioridade, adiciona-o ao conjunto de visitados e relaxa todas as arestas que saem de $u$. \citeonline{dasgupta} destacam que a eficiência do Dijkstra depende da estrutura de dados utilizada. Com um Heap Binário, a complexidade é $O((V+E)\log V)$. \citeonline{cormen} apontam que, ao utilizar um Heap de Fibonacci, a complexidade amortizada pode ser reduzida para $O(E+VlogV)$, o que é crucial para grafos densos.

\begin{algorithm}[H]
    \DontPrintSemicolon
    \SetAlgoLined

    \Fn{Dijkstra($G, w, s$)}{
        Initialize-Single-Source($G, s$)\;
        $S \leftarrow \emptyset$\;
        $Q \leftarrow G.V$\;
        \While{$Q \neq \emptyset$}{
            $u \leftarrow \text{Extract-Min}(Q)$\;
            $S \leftarrow S \cup \{u\}$\;
            \ForEach{vertex $v \in G.Adj[u]$}{
                Relax($u, v, w$)\;
            }
        }
    }
    \caption{Algoritmo de Dijkstra \cite{cormen}}
\end{algorithm}

\subsubsection{Algoritmo de Bellman-Ford}

Quando o grafo contém arestas com pesos negativos, o algoritmo de Dijkstra não garante a corretude. Nesses casos, utiliza-se o Algoritmo de Bellman-Ford. Segundo \citeonline{manber}, este algoritmo baseia-se em programação dinâmica, relaxando todas as arestas do grafo $|V|-1$ vezes.

A propriedade fundamental do Bellman-Ford é sua capacidade de detectar ciclos negativos. Se, após $|V|-1$ iterações, ainda for possível relaxar alguma aresta, o grafo contém um ciclo com peso total negativo acessível a partir da fonte, o que implica que $\delta(s,v)=-\infty$ para alguns vértices \cite{kleinberg}. Sua complexidade é $O(VE)$.

\begin{algorithm}[H]
    \DontPrintSemicolon
    \SetAlgoLined

    \Fn{Bellman-Ford($G, w, s$)}{
        Initialize-Single-Source($G, s$)\;
        \For{$i = 1$ \KwTo $|G.V| - 1$}{
            \ForEach{edge $(u, v) \in G.E$}{
                Relax($u, v, w$)\;
            }
        }
        \ForEach{edge $(u, v) \in G.E$}{
            \If{$v.d > u.d + w(u, v)$}{
                \KwRet \KwFalse\;
            }
        }
        \KwRet \KwTrue\;
    }
    \caption{Algoritmo de Bellman-Ford \cite{cormen}}
\end{algorithm}

\subsubsection{Algoritmo de Floyd-Warshall}

Enquanto Dijkstra e Bellman-Ford resolvem o problema de fonte única (Single-Source), o algoritmo de Floyd-Warshall resolve o problema de caminhos mais curtos entre todos os pares de vértices (All-Pairs).

\citeonline{aho} explicam que o algoritmo utiliza programação dinâmica baseada na numeração dos vértices. Seja $d_{ij}^{(k)}$ o peso do caminho mais curto de $i$ a $j$ utilizando apenas vértices do conjunto ${1,2,\dots,k}$ como intermediários. A recorrência é dada por:

\begin{equation}
    d_{ij}^{(k)} = \min(d_{ij}^{(k-1)}, d_{ik}^{(k-1)} + d_{kj}^{(k-1)})
\end{equation}

A complexidade temporal é $\theta(V^3)$. \citeonline{jungnickel} observa que, para grafos densos, este método é geralmente mais eficiente do que executar o algoritmo de Bellman-Ford a partir de cada vértice.

\begin{algorithm}[H]
    \DontPrintSemicolon
    \SetAlgoLined

    \Fn{Floyd-Warshall($W$)}{
        $n \leftarrow W.rows$\;
        $D^{(0)} \leftarrow W$\;
        \For{$k = 1$ \KwTo $n$}{
            \KwLet $D^{(k)} = (d_{ij}^{(k)})$ be a new $n \times n$ matrix\;
            \For{$i = 1$ \KwTo $n$}{
                \For{$j = 1$ \KwTo $n$}{
                    $d_{ij}^{(k)} \leftarrow \min \left( d_{ij}^{(k-1)}, d_{ik}^{(k-1)} + d_{kj}^{(k-1)} \right)$\;
                }
            }
        }
        \KwRet $D^{(n)}$\;
    }
    \caption{Algoritmo de Floyd-Warshall \cite{cormen}}
\end{algorithm}

% 2.7.5. Reponderação e Algoritmo de Johnson

% Uma técnica avançada, relevante para algoritmos de fluxo de custo mínimo, é a reponderação de arestas para eliminar pesos negativos sem alterar os caminhos mais curtos. O Algoritmo de Johnson combina Bellman-Ford e Dijkstra.

% Conforme descrito por Cormen et al. (2009) e Ahuja et al. (1993), utiliza-se uma função de potencial h:V→R tal que o novo peso da aresta seja w^(u,v)=w(u,v)+h(u)−h(v)≥0. Se o grafo não contiver ciclos negativos, é possível calcular tais potenciais (usando Bellman-Ford uma vez) e subsequentemente rodar Dijkstra para cada vértice, obtendo uma complexidade assintótica melhor que Floyd-Warshall para grafos esparsos: O(V2logV+VE).

\subsection{Redes de fluxo em grafos}

A teoria dos fluxos em redes é um ramo fundamental da otimização combinatória que modela problemas de transporte de recursos através de um sistema conectado. Esta seção define os conceitos preliminares e apresenta os teoremas centrais que sustentam os algoritmos utilizados neste trabalho.

\subsubsection{Definições básicas}

Uma rede de fluxo é definida por um grafo direcionado $G = (V, E)$, onde $V$ é o conjunto de vértices e $E$ é o conjunto de arestas. Distinguem-se dois vértices especiais: a \textbf{fonte} $s\in V$. que produz o fluxo, e o \textbf{sorvedouro} $t\in V$, que o consome. Para cada aresta $(u, v) \in E$, associa-se uma capacidade não-negativa $c(u,v) \geq o$, que limita a quantidade de fluxo que pode atravessar aquela aresta. Um \textbf{fluxo} em $G$ é uma função $f \space : \space E \rightarrow \mathbb{R}^+$ que satisfaz as seguintes propriedades \cite{cormen}:

\begin{itemize}
    \item \textbf{Restrição de capacidade:} O fluxo em uma aresta não pode exceder a sua capacidade.
    \begin{equation}
        \label{eq:capacity_restraint}
        0 \leq f(u,v) \leq c(u,v), \qquad \forall(u,v) \in E
    \end{equation}
    \item \textbf{Conservação de fluxo:} para todo vértice $v \in V - \{s,t\}$, a quantidade total de fluxo que entra deve ser igual à que sai
    \begin{equation}
        \sum_{v \in V}f(v,u) = \sum_{v \in V}f(u,v)
    \end{equation}
\end{itemize}

\subsubsection{O problema do fluxo máximo}

O \textbf{problema do fluxo máximo} consiste em encontrar um fluxo $f$ tal que $|f|$ seja o maior possível. Para entender a limitação desse fluxo, precisamos entender os conceitos de corte, redes residuais e caminhos aumentantes.

\paragraph{Corte}

Um \textbf{corte} $(S,T)$ é uma partição do conjunto de vértices $V$ em dois subconjuntos disjuntos $S$ e $T$, tal que $s\in S$ e $t\in T$. A \textbf{capacidade do corte}, denotada por $c(S,T)$, é a soma das capacidades das arestas que partem de $S$ para $T$ \cite{cormen}:

\begin{equation}
    c(S,T) = \sum_{u\in S,v\in T}c(u,v)
\end{equation}

O \textbf{corte mínimo} de uma rede $N$ é um corte cuja capacidade é a menor entre todos os cortes possíveis nessa rede \cite{cormen}. Isso nos leva ao \textbf{teorema do fluxo máximo e corte mínimo (Max-flow Min-cut)}

\begin{quotation}
    \textit{O valor máximo do fluxo de uma rede $N$ é igual à capacidade do corte mínimo em $N$ \cite{jungnickel}}
\end{quotation}

Este teorema é fundamental para a compreensão e corretudo dos algoritmos de caminhhos aumentantes.

\paragraph{Redes residuais}

A maioria dos algoritmos de resolução de problemas de fluxo máximo se baseia no conceito de redes residuais. Uma \textbf{rede residual} $G_f$ consiste em arestas com capacidades que representam como podemos alterar o fluxo das arestas em $G$. A capacidade das arestas da rede residual $G_f$ corresponde à diferença entre a capacidade da aresta original de $G$ e o fluxo $f$ que está passando pela aresta \cite{cormen}. Ou seja:

\begin{equation}
    c_f(u,v) = c(u,v) - f(u,v)
\end{equation}

Quando a aresta $(u,v)$ possui fluxo igual à sua capacidade, então $c_f(u,v) = 0$ e a aresta não é incluída no grafo residual $G_f$

\paragraph{Caminhos aumentantes}
\label{p:ford_fulkerson}

Dado um grafo $G=(V,E)$ e um fluxo $f$, um \textbf{caminho aumentante} é um caminho de $s$ para $t$ na rede residual $G_f$. Por definição, podemos aumentar o fluxo em uma aresta $(u,v)$ de um caminho aumentante em até $c_f(u,v)$ sem violar a capacidade da aresta original em $G$ \cite{cormen}.

O método de Ford-Fulkerson, que é base dos principais algoritmos de resolução do problema do fluxo máximo, iterativamente encontra um caminho aumentante, calcula sua capacidade residual e adiciona esse fluxo à solução atual até que não existam mais caminhos aumentantes \cite{ahuja}.

\begin{algorithm}[H]
    \DontPrintSemicolon
    \SetAlgoLined

    \Fn{FORD-FULKERSON($G, s, t$)}{
        \ForEach{edge $(u, v) \in G.E$}{
            $(u, v).f \leftarrow 0$\;
        }
        
        \While{there exists a path $p$ from $s$ to $t$ in the residual network $G_f$}{
            $c_f(p) \leftarrow \min \{ c_f(u, v) : (u, v) \text{ is in } p \}$\;
            
            \ForEach{edge $(u, v) \in p$}{
                \eIf{$(u, v) \in E$}{
                    $(u, v).f \leftarrow (u, v).f + c_f(p)$\;
                }{
                    $(v, u).f \leftarrow (v, u).f - c_f(p)$\; 
                }
            }
        }
    }
    
    \caption{Ford-Fulkerson Algorithm}
    \label{alg:ford_fulkerson}
\end{algorithm}

\subsubsection{Fluxo de custo mínimo}

Uma outra variação dos problemas em redes de fluxo em grafos é o problema do fluxo de custo mínimo. Essa variação considera, além das capacidades disponíveis, uma dimensão econômica. Seja $G=(V, E)$ uma rede direcionada one cada aresta $(u,v)$ possui uma capacidade $c(u,v)$ e um custo unitário $w(u,v)$. O objetivo é transmitir uma quantidade de fluxo pré-determinada $F$ da fonte $s$ ao sorvedouro $t$ com o menor custo total possível.

A formulação linear do problema é dada por \citeonline{ahuja}. O primeiro objetivo é minimizar o custo total do fluxo, que segue a fórmula:

\begin{equation}
    \text{Minimizar o custo do fluxo:}\quad z(f) = \sum_{(u,v)\in A} w(u,v) \cdot f(u,v)
\end{equation}

A função a seguir assegura que o nó fonte $s$ gere exatamente a demanda $F$, e que o nó sorvedouro $t$ absorva-a integralmente. 

\begin{equation}
    \sum_{i,j\in V}f(i,j) - \sum_{k,i\in V}f(k,i) = 
        \begin{cases}
            F  & \text{se } i = s \\
            -F & \text{se } j = t \\
            0  & \text{caso contrário}
        \end{cases}
\end{equation}

Além desta, é importante lembrar da equação \ref{eq:capacity_restraint}, que define que o fluxo em uma aresta não pode exceder a sua capacidade.

\paragraph{Condições de otimalidade e ciclos negativos}

O critério fundamental para verificar se uma solução é ótima é o \textbf{teorema dos ciclos negativos}:

\begin{quotation}
    Um fluxo $f$ é uma solução viável para o problema do fluxo de custo mínimo se e somente se ele satisfaz a condição de otimalidade de ciclo negativo: a rede resiual $G(f)$ não contém um ciclo e custo negativo \cite{ahuja}
\end{quotation}

Este teorema é a base dos algoritmos de \textbf{cancelamento de ciclos} (Cycle Canceling), que iterativamente identificam ciclos negativos na rede residual e enviam fluxo através deles para reduzir o custo total da função objetivo até que nenhum ciclo negativo reste.

\paragraph{Potenciais de vértices e custos reduzidos}

Embora o cancelamento de ciclos seja teoricamente sólido, a detecção de ciclos negativos (via Bellman-Ford) pode ser computacionalmente custosa. Para permitir o uso de algoritmos mais eficientes como o de Dijkstra, a literatura introduz o conceito de \textbf{potenciais de vértices}, baseando-se na teoria de dualidade da programação linear.

A ideia é manter um \textbf{potencial} para cada vértice $v$ do grafo, chamado de $h(v)$, com valor inicial 0. Definimos o \textbf{custo reduzido} de uma aresta $(u,v)$ como $\gamma'(u,v) = \gamma(u,v) + h(u) - h(v)$. Se os potenciais forem escolhidos de forma adequada, todos os custos reduzidos podem ser garantidos como não-negativos, permitindo o uso do algoritmo de Dijkstra. Além disso, é garantido que os caminhos de custo mínimo no grafo original e no grafo com custos reduzidos são os mesmos. \cite{jungnickel}.

O conceito de potenciais e de custo reduzido será utilizado em algoritmos como o de caminhos sucessivos para resolução do problema de atribuição na seção \ref{p:reduction_to_minimal_flow}.

% ==================================================================================================================================
% 2.2. Problema Geral do Emparelhamento
% ==================================================================================================================================
\subsection{Problema Geral do Emparelhamento}

O problema de \textbf{matching} (emparelhamento) em grafos é um problema fundamental na otimização combinatória

Um \textbf{matching} M em um grafo não-direcionado $G=(V,E)$ é um subconjunto de arestas $M \subseteq E$ tal que
nenhum par de arestas em $M$ compartilha um vértice comum \cite{kleinberg}. Em outras palavras, cada nó aparece em no máximo uma aresta de $M$.

\begin{itemize}
    \item Um vértice é chamado \textbf{coberto} (\textit{matched}) se for incidente a uma aresta em $M$; caso contrário, 
    é \textbf{descoberto} (\textit{unmatched} ou \textit{exposed}) \cite{cormen}
    \item Um \textbf{matching de cardinalidade máxima} (Maximum Matching) é um matching com o maior número possível 
    de arestas \cite{cormen}. A cardinalidade máxima de um matching é denotada por $v(G)$ \cite{schrijver}
    \item Um \textbf{matching perfeito} é um matching que cobre todos os vértices do grafo \cite{kleinberg}
    \item  O \textbf{problema de matching ponderado} (Weighted Matching Problem) envolve encontrar um matching para o qual a 
    soma dos pesos das arestas é máxima. Em um grafo ponderado $G=(V,E,w)$, busca-se um $M\subseteq E$ que maximize $w(M)$ \cite{matching_gpus_sc24}
\end{itemize}

% ---------------------------------------------------------------------------------------------------------------------------------
% 2.2.1. Matching em grafos bipartidos
% ---------------------------------------------------------------------------------------------------------------------------------
\subsubsection{Matching em grafos bipartidos}

O problema de \textbf{Matching Bipartido} é o caso clássico de encontrar um matching de cardinalidade máxima em um grafo 
bipartido $G=(V,E)$, onde $V=X\cup Y$ \cite{kleinberg}

\begin{itemize}
    \item O matching em grafos bipartidos pode modelar situações de atribuição, como associar empregos ($X$) a máquinas ($Y$), 
    ou professores ($X$) a cursos ($Y$), onde uma aresta indica uma capacidade de atribuição \cite{kleinberg}
    \item O problema de matching ponderado em grafos bipartidos é equivalente ao \textbf{problema de atribuição \cite{lawler}}, 
    que historicamente motivou o desenvolvimento do \textbf{método Húngaro} \cite{schrijver}
\end{itemize}

% ---------------------------------------------------------------------------------------------------------------------------------
% 2.2.2. Formulação geral (emparelhamento e otimização)
% ---------------------------------------------------------------------------------------------------------------------------------
\subsubsection{Formulação geral (emparelhamento e otimização)}

Em um contexto mais amplo, o emparelhamento de grafos pode ser formalizado como um problema de otimização que 
busca maximizar a compatibilidade entre dois grafos $G$ e $G'$ \cite{learning_graph_matching}.

\begin{itemize}
    \item  O problema de graph matching é frequentemente abordado como um \textbf{problema de atribuição quadrática (QAP)}. 
    Essa formulação busca maximizar uma função objetivo que combina termos de compatibilidade unária (nó-nó, $c_{ii'}$) 
    e compatibilidade par a par (aresta-aresta, $d_{ii'jj'}$), sujeito a restrições de atribuição binária ($y_{ii'} \in {0,1}$). 
    O termo quadrático codifica a preservação das relações (arestas) entre os nós \cite{learning_graph_matching}
    \item Para grafos bipartidos, a determinação de um matching máximo pode ser resolvida de maneira eficiente e está intimamente 
    ligada a problemas de \textit{network flow} (fluxo em redes) \cite{kleinberg}
\end{itemize}

\input{capitulos/3_taxonomia.tex}
%!TeX root = ../main.tex


% ================================================================================================
% 4. Problemas de emparelhamento em grafos bipartidos
% ================================================================================================
\section{Problemas de emparelhamento em grafos bipartidos}

% ------------------------------------------------------------------------------------------------
% 4.1. Emparelhamento de cardinalidade máxima
% ------------------------------------------------------------------------------------------------
\subsection{Emparelhamento de cardinalidade máxima}

\subsubsection{Descrição do problema}

Dado um grafo não-direcionado $G=(V,E)$, o problema de emparelhamento de cardinalidade máxima (MCM) busca encontrar um emparelhamento $M \subseteq E$ tal que o número de arestas em $M$ seja maximizado, ou seja, $|M|$ é o maior entre todos os emparelhamentos possíveis em $G$.

Um caso especial, mas importante, do MCM é o \textbf{Emparelhamento perfeito}, que é um emparelhamento onde todos os vértices do grafo são cobertos por exatamente uma aresta do emparelhamento. Para que um emparelhamento perfeito exista, o grafo deve ter um número par de vértices e o emparelhamento máximo deve ter cardinalidade igual a $|V|/2$ \cite{jungnickel}.

\subsubsection{Propriedades}

Antes de continuar para os algoritmos e soluções, é necessário entender algumas propriedades importantes

\paragraph{Emparelhamento máximo vs maximal}

Um emparelhamento máximo é aquele que contém o maior número possível de arestas, enquanto um emparelhamento maximal é aquele que não pode ser aumentado adicionando mais arestas, mas não necessariamente é o maior possível \cite{manber}.

\paragraph{Caminhos aumentantes}

Um caminho aumentante $P$ em um grafo $G$ é chamado de caminho aumentante se $P$ tem tamanho ímpar, começa e termina em vértices livres (não cobertos por nenhuma aresta do emparelhamento atual) e alterna entre arestas que pertencem a $M$ e arestas que não pertencem a $M$. A existência de um caminho aumentante indica que o emparelhamento atual não é máximo \cite{schrijver}

\subsubsection{Algoritmos exatos}
\label{sec:algos_exatos_emp_card_max}

Por ser em um ambiente mais restrito, o problema de emparelhamento de cardinalidade máxima tem diversas soluções em tempo polinomial. A seguir listadas algumas delas:

\paragraph{Algoritmo do caminho aumentante}

O algoritmo de caminho aumentante se baseia em um lema proposto por Claude Berge, que afirma que um grafo G é maximo se e somente se não existe nenhum caminho aumentante em G. Este algoritmo é uma implementação direta deste lema, onde se busca repetidamente por caminhos aumentantes e se aumenta o emparelhamento até que nenhum caminho aumentante possa ser encontrado \cite{lawler}.

Começamos com um emparelhamento vazio $M=\emptyset$. Escolhemos um vértice livre $u$ no lado esquerdo do grafo bipartido e tentamos encontrar um caminho aumentante $P$ que começa em $u$ e termina em um vértice livre no lado direito do grafo, respeitando a regra da alternância entre arestas em $M$ e arestas fora de $M$. Se um caminho aumentante for encontrado, atualizamos o emparelhamento $M$ invertendo as arestas ao longo do caminho $P$. Repetimos esse processo até que não seja possível encontrar mais caminhos aumentantes. O emparelhamento resultante será o emparelhamento de cardinalidade máxima \cite{halim}.

Em seguida o algoritmo em pseudo-código:

\begin{lstlisting}[language=C, caption=Algoritmo de caminhos aumentantes \cite{halim}]
// global variables
vi match, vis; // ('vi' is an alias for vector<int>)

int Aug(int l) { // return 1 if an augmenting path is found
    if (vis[l]) return 0; // return 0 otherwise
    vis[l] = 1;
    for (int j = 0; j < (int)AdjList[l].size(); j++) {
        int r = AdjList[l][j]; // edge weight not needed -> vector<vi> AdjList
        if (match[r] == -1 || Aug(match[r])) {
            match[r] = l; return 1; // found 1 matching
    } }
    return 0; // no matching
}
// inside int main()
    // build unweighted bipartite graph with directed edge left->right set
    int MCBM = 0;
    match.assign(V, -1); // V is the number of vertices in bipartite graph
    for (int l = 0; l < n; l++) { // n = size of the left set
        vis.assign(n, 0); // reset before each recursion
        MCBM += Aug(l);
    }
    printf("Found %d matchings\n", MCBM);
\end{lstlisting}

Como o algoritmo repete a busca por caminhos aumentantes para cada vértice do lado esquerdo do grafo, o tempo de execução total do algoritmo é $O(VE)$, onde $V$ é o número de vértices e $E$ é o número de arestas no grafo bipartido \cite{halim}.

\paragraph{Redução ao problema de fluxo máximo}

O problema de emparelhamento de cardinalidade máxima em grafos bipartidos pode ser eficientemente resolvido através de uma redução ao problema de fluxo máximo em redes. A ideia central é construir uma rede de fluxo a partir do grafo bipartido original, onde cada aresta do grafo bipartido é convertida em uma aresta com capacidade unitária na rede de fluxo. \cite{lawler}

Primeiro, adicionamos um vértice fonte $s$ e um vértice sumidouro $t$ à rede. Conectamos o vértice fonte $s$ a todos os vértices do conjunto esquerdo do grafo bipartido com arestas de capacidade 1. Em seguida, conectamos todos os vértices do conjunto direito do grafo bipartido ao vértice sumidouro $t$, também com arestas de capacidade 1. As arestas entre os conjuntos esquerdo e direito do grafo bipartido são mantidas na rede de fluxo, cada uma com capacidade 1 \cite{halim}.

Com isso, temos agora um grafo de fluxo onde o objetivo é encontrar o fluxo máximo do vértice fonte $s$ para o vértice sumidouro $t$. O valor do fluxo máximo encontrado nesta rede corresponde ao tamanho do emparelhamento máximo no grafo bipartido original. \cite{lawler}. Para encontrar o fluxo máximo, podemos utilizar algoritmos clássicos como o de Edmonds-Karp.

O algoritmo de Edmonds-Karp é o padrão para resolver o problema de fluxo máximo, utilizando buscas em largura (BFS) para encontrar caminhos aumentantes na rede residual. A seguir, o pseudo-código do algoritmo:

\begin{lstlisting}[language=C, caption=Algoritmo de Edmonds-Karp \cite{halim}]
int res[MAX_V][MAX_V], mf, f, s, t; // global variables
vi p; // p stores the BFS spanning tree from s

void augment(int v, int minEdge) { // traverse BFS spanning tree from s->t
    if (v == s) { f = minEdge; return; } // record minEdge in a global var f
    else if (p[v] != -1) { 
        augment(p[v], min(minEdge, res[p[v]][v]));
        res[p[v]][v] -= f; res[v][p[v]] += f; 
    } 
}

// inside int main(): set up 'res', 's', and 't' with appropriate values
    mf = 0; // mf stands for max_flow

    while (1) { // O(VE^2) (actually O(V^3 E) Edmonds Karp's algorithm
        f = 0;

        // run BFS, compare with the original BFS shown in Section 4.2.2
        vi dist(MAX_V, INF); dist[s] = 0; queue<int> q; q.push(s);
        p.assign(MAX_V, -1); // record the BFS spanning tree, from s to t!

        while (!q.empty()) {
            int u = q.front(); q.pop();
            if (u == t) break; // immediately stop BFS if we already reach sink t

            for (int v = 0; v < MAX_V; v++) // note: this part is slow
                if (res[u][v] > 0 && dist[v] == INF)
                    dist[v] = dist[u] + 1, q.push(v), p[v] = u; // 3 lines in 1!
        }

        augment(t, INF); // find the min edge weight 'f' in this path, if any
        if (f == 0) break; // we cannot send any more flow ('f' = 0), terminate
        mf += f; // we can still send a flow, increase the max flow!
    }
    printf("%d\n", mf); 
\end{lstlisting}

Outra opção para resolver o problema de fluxo máximo é o algoritmo de Dinic, que é mais eficiente em muitos casos práticos, especialmente em grafos densos. O algoritmo de Dinic utiliza uma combinação de buscas em largura (BFS) para construir níveis na rede residual e buscas em profundidade (DFS) para encontrar caminhos aumentantes dentro desses níveis. O tempo de execução do algoritmo de Dinic é $O(E \sqrt{V})$ para grafos gerais, tornando-o uma escolha preferida para muitos problemas de fluxo máximo \cite{halim}.

Este trabalho não tem uma implementação do algoritmo de Dinic. Ao invés disso, será apresentado o algoritmo de Hopcroft-Karp, que é uma variação especializada do algoritmo de Dinic para o problema de emparelhamento em grafos bipartidos, oferecendo uma solução mais eficiente para este caso específico.

\paragraph{Algoritmo de Hopcroft-Karp}

O algoritmo de Hopcroft-Karp é um algoritmo eficiente para encontrar o emparelhamento máximo em grafos bipartidos. Ele melhora o desempenho do algoritmo de caminhos aumentantes ao encontrar múltiplos caminhos aumentantes em cada iteração, em vez de apenas um. O tempo de execução do algoritmo de Hopcroft-Karp é $O(E \sqrt{V})$, tornando-o significativamente mais rápido do que o algoritmo de caminhos aumentantes simples, especialmente em grafos grandes \cite{halim}.

O algoritmo consiste em duas fases principais: a fase de construção de níveis e a fase de busca de caminhos aumentantes. Na fase de construção de níveis, uma busca em largura (BFS) é realizada a partir dos vértices livres no lado esquerdo do grafo bipartido para construir uma camada de níveis que ajuda a identificar os caminhos aumentantes mais curtos. Na fase de busca de caminhos aumentantes, uma busca em profundidade (DFS) é realizada para encontrar todos os caminhos aumentantes possíveis dentro da camada de níveis construída na fase anterior. Esses caminhos são então usados para aumentar o emparelhamento \cite{halim}.

A seguir, o pseudo-código do algoritmo de Hopcroft-Karp:

\begin{lstlisting}[language=C, caption=Algoritmo de Hopcroft-Karp (implementação própria)]
ALGORITMO Hopcroft-Karp(G, U, V):
    Para cada u em U: PairU[u] = NIL
    Para cada v em V: PairV[v] = NIL
    Matching = 0

    Enquanto BFS() for verdadeiro:
        Para cada u em U:
            Se PairU[u] == NIL:
                Se DFS(u) for verdadeiro:
                    Matching = Matching + 1
    
    Retornar Matching

---------------------------------------------------------

FUNCAO BFS():
    Fila Q = vazia
    Para cada u em U:
        Se PairU[u] == NIL:
            Dist[u] = 0
            Q.push(u)
        SenAo:
            Dist[u] = INFINITO
    
    Dist[NIL] = INFINITO

    Enquanto Q nAo estiver vazia:
        u = Q.pop()
        
        // Se a distAncia atual for menor que a distancia para o NIL, 
        // continuamos procurando. Se for maior, ja achamos um caminho mais curto antes.
        Se Dist[u] < Dist[NIL]:
            Para cada v adjacente a u:
                // Se v ja tem par, verificamos a distancia desse par
                Se Dist[PairV[v]] == INFINITO:
                    Dist[PairV[v]] = Dist[u] + 1
                    Q.push(PairV[v])
    
    // Retorna verdadeiro se alcancamos o NIL (ou seja, achamos um caminho aumentante livre)
    Retornar Dist[NIL] != INFINITO

---------------------------------------------------------

FUNCAO DFS(u):
    Se u != NIL:
        Para cada v adjacente a u:
            // So seguimos se o vizinho estiver na proxima "camada" valida (distancia + 1)
            Se Dist[PairV[v]] == Dist[u] + 1:
                Se DFS(PairV[v]) for verdadeiro:
                    PairV[v] = u
                    PairU[u] = v
                    Retornar VERDADEIRO
        
        // Otimizacao: Se nao achou caminho por u, marca como infinito para nao tentar de novo nesta fase
        Dist[u] = INFINITO
        Retornar FALSO
    
    Retornar VERDADEIRO
\end{lstlisting}


% ----------------------------------------------------------------------------------------------
% 3.2.1. Problema de atribuição (Assignment Problem)
% ----------------------------------------------------------------------------------------------
\subsection{Problema de atribuição (Assignment Problem)}

\subsubsection{Descrição do problema}

O problema de atribuição, também conhecido como problema do emparelhamento ponderado, consiste em encontrar uma combinação ótima de atribuições entre dois conjuntos disjuntos, minimizando o custo total associado a essas atribuições. Exemplo: Considere $N$ trabalhadores e $N$ tarefas, onde cada trabalhador pode ser designado a exatamente uma tarefa, e cada tarefa deve ser atribuída a exatamente um trabalhador. O custo de atribuir o trabalhador i à tarefa j é representado por uma matriz de custos $W = [w_{ij}]$ \cite{lawler}. O objetivo é encontrar um conjunto de atribuições que minimize o custo total.

\subsubsection{Propriedades}

\subsubsection{Algoritmos exatos}

% \paragraph{Método Húngaro}
\paragraph{Método Húngaro}

O método Húngaro é um algoritmo eficiente para resolver o problema de atribuição em tempo polinomial. Ele foi desenvolvido por Harold Kuhn em 1955 e é baseado no trabalho de Dénes Kőnig e Jenő Egerváry sobre emparelhamentos em grafos bipartidos \cite{jungnickel}.

O método se baseia em manter um conjunto de potenciais para os vértices dos dois conjuntos do grafo bipartido, sendo $u_i$ o potencial do vértice i no conjunto esquerdo e $v_j$ o potencial do vértice j no conjunto direito. A regra que o algoritmo mantém é que para cada aresta $(i,j)$, a soma dos potenciais deve ser maior ou igual ao custo da aresta, ou seja, $u_i + v_j \geq w_{ij}$. 

Cria-se um subgrafo de igualdade $H_{u,v}$ contendo apenas as arestas onde a soma dos potenciais é igual ao custo da aresta, ou seja, $u_i + v_j = w_{ij}$. O algoritmo então tenta encontrar um emparelhamento máximo neste subgrafo de igualdade. Se o emparelhamento encontrado cobre todos os vértices do conjunto esquerdo, então ele é ótimo para o problema de atribuição original. Caso contrário, o algoritmo ajusta os potenciais para criar novas arestas de igualdade e repete o processo até encontrar um emparelhamento máximo que cubra todos os vértices do conjunto esquerdo \cite{jungnickel}.

Em seguida o pseudo-código do método Húngaro:

\begin{algorithm}[H]
    \DontPrintSemicolon
    \SetAlgoLined

    \Fn{HUNGARIAN($n, w; mate$)}{
        
        % Lines 1-3
        \lFor{$v \in V$}{ $mate(v) \leftarrow 0$ }
        \lFor{$i = 1$ \KwTo $n$}{ $u_i \leftarrow \max \{ w_{ij} : j=1,\dots,n \}; v_i \leftarrow 0$ }
        $nrex \leftarrow n$\;

        % Line 4: While Loop
        \While{$nrex \neq 0$}{
            
            % Line 5
            \lFor{$i = 1$ \KwTo $n$}{ $m(i) \leftarrow \text{false}; p(i) \leftarrow 0; \delta_i \leftarrow \infty$ }
            $aug \leftarrow \text{false}; Q \leftarrow \{ i \in S : mate(i) = 0 \}$\;
            
            % Line 7: Repeat Loop
            \Repeat{$aug = \text{true}$}{
                remove an arbitrary vertex $i$ from $Q$; $m(i) \leftarrow \text{true}; j \leftarrow 1$\;
                
                % Line 9: Inner While
                \While{$aug = \text{false}$ \KwAnd $j \leq n$}{
                    \If{$mate(i) \neq j'$}{
                         \If{$u_i + v_j - w_{ij} < \delta_j$}{
                            $\delta_j \leftarrow u_i + v_j - w_{ij}; p(j) \leftarrow i$\;
                            \If{$\delta_j = 0$}{
                                \eIf{$mate(j') = 0$}{
                                    AUGMENT($mate, p, j'; mate$)\;
                                    $aug \leftarrow \text{true}; nrex \leftarrow nrex - 1$\;
                                }{
                                    $Q \leftarrow Q \cup \{ mate(j') \}$\;
                                }
                            }
                        }
                    }
                    $j \leftarrow j + 1$\;
                }
                
                % Line 24: Check if augmentation failed and Q is empty
                \If{$aug = \text{false}$ \KwAnd $Q = \emptyset$}{
                    $J \leftarrow \{ i \in S : m(i) = \text{true} \}; K \leftarrow \{ j' \in T : \delta_j = 0 \}$\;
                    $\delta \leftarrow \min \{ \delta_j : j' \in T \setminus K \}$\;
                    \lFor{$i \in J$}{ $u_i \leftarrow u_i - \delta$ }
                    \lFor{$j' \in K$}{ $v_j \leftarrow v_j + \delta$ }
                    \lFor{$j' \in T \setminus K$}{ $\delta_j \leftarrow \delta_j - \delta$ }
                    $X \leftarrow \{ j' \in T \setminus K : \delta_j = 0 \}$\;
                    
                    \eIf{$mate(j') \neq 0$ for all $j' \in X$}{
                        \lFor{$j' \in X$}{ $Q \leftarrow Q \cup \{ mate(j') \}$ }
                    }{
                        choose $j' \in X$ with $mate(j') = 0$\;
                        AUGMENT($mate, p, j'; mate$)\;
                        $aug \leftarrow \text{true}; nrex \leftarrow nrex - 1$\;
                    }
                }
            }
        }
    }
    \caption{Algoritmo Húngaro \cite{jungnickel}}
    \label{alg:hungarian}
\end{algorithm}

\begin{algorithm}[H]
    \DontPrintSemicolon
    \SetAlgoLined

    \Fn{AUGMENT($mate, p, j'; mate$)}{
        \Repeat{$next = 0$}{
            $i \leftarrow p(j); mate(j') \leftarrow i; next \leftarrow mate(i); mate(i) \leftarrow j'$\;
            \If{$next \neq 0$}{
                $j' \leftarrow next$\;
            }
        }
    }
    \caption{Método AUGMENT \cite{jungnickel}}
    \label{alg:augment}
\end{algorithm}

\paragraph{Jonker-Volgenant} 

O algoritmo de Jonker-Volgenant é uma evolução otimizada para a resolução do problema de atribuição linear, proposto por Roy Jonker e Anton Volgenant em 1987. Ele foi desenvolvido para superar o desempenho prático do método Húngaro, especialmente em grafos densos, combinando a teoria de dualidade com estratégias de inicialização e busca mais eficientes \cite{jonker}.

Assim como o método Húngaro, este algoritmo baseia-se na manutenção de potenciais duais ($u_i$ e $v_j$) e respeita as condições de folga complementar. No entanto, sua principal distinção conceitual é o uso da estratégia de Caminho Aumentante Mais Curto (Shortest Augmenting Path). Ao invés de construir emparelhamentos máximos em fases distintas, o algoritmo foca em encontrar o caminho de custo mínimo que conecta uma linha não atribuída a uma coluna livre.

O processo inicia com heurísticas avançadas (conhecidas como redução de colunas e transferência de redução) para resolver rapidamente as atribuições triviais. Para os vértices restantes, o algoritmo realiza uma busca similar ao algoritmo de Dijkstra: ele explora o grafo para encontrar o caminho aumentante mais curto, atualizando os potenciais duais simultaneamente durante a busca. Isso garante que, a cada iteração, o emparelhamento seja aumentado pelo menor custo possível até que a solução ótima completa seja atingida.

A implementação deste algoritmo é bastante complexa, e portanto não será detalhada aqui. No entanto, sua eficiência prática o torna uma escolha preferida para muitos problemas de atribuição em aplicações reais, especialmente quando comparado ao método Húngaro tradicional \cite{jungnickel}.

% TODO: adicionar pseudo-código do algoritmo Jonker-Volgenant
% "A Shortest Augmenting Path Algorithm for Dense and Sparse Linear Assignment Problems"

\paragraph{Redução para problema de fluxo de custo mínimo}

Similarmente ao que foi feito para o problema de emparelhamento de cardinalidade máxima, o problema de atribuição pode ser resolvido através de uma redução ao problema de fluxo de custo mínimo. Assim como para o problema anterior, essa redução é feita criando dois novos vértices: um vértice fonte $s$ e um vértice sorvedouro $t$. Em seguida, são adicionadas arestas do vértice fonte $s$ para cada vértice de um dos conjuntos do grafo bipartido, com capacidade 1 e custo 0. Também são adicionadas arestas de cada vértice do outro conjunto para o vértice sorvedouro $t$, também com capacidade 1 e custo 0. Finalmente, As arestas do grafo bipartido original são adicionadas com capacidade 1 e custo igual ao custo de atribuição correspondente.

Para resolver problemas de fluxo podemos utilizar algoritmos como o de caminhos de custo mínimo sucessivos (Successive Shortest Paths) ou o algoritmo de cancelamento de ciclos (Cycle canceling) \cite{schrijver}.

% TODO: Incluir também Out-of-kilter, Simplex, e cost-scaling

O primeiro algoritmo, o de caminhos de custo mínimo sucessivos, foi inicialmente proposto por Jewell, Busacker e Gowen e é uma adaptação do algoritmo de Ford-Fulkerson para incorporar o custo das arestas. Em poucas palavras, ele encontra o caminho de $s$ para $t$ de custo mínimo, envia o máximo fluxo possível ao longo desse caminho, e repete esse processo até que não seja mais possível aumentar o fluxo sem violar as capacidades das arestas \cite{schrijver}.

\begin{algorithm}[H]
    \DontPrintSemicolon
    \SetAlgoLined

    \Fn{OPTFLOW($G, c, s, t, \gamma, v; f, \text{sol}$)}{
        % Line 1: Single-line for loop
        \lFor{$e \in E$}{ $f(e) \leftarrow 0$ }
        
        % Line 2
        $\text{sol} \leftarrow \text{true}, \text{val} \leftarrow 0$\;

        % Line 3: While loop
        \While{$\text{sol} = \text{true}$ \KwAnd $\text{val} < v$}{
            
            % Line 4: Descriptive text
            construct the auxiliary network $N' = (G', c', s, t)$ with cost function $\gamma'$\;

            % Line 5: If-Else structure
            \eIf{$t$ is not accessible from $s$ in $G'$}{
                % Line 6
                $\text{sol} \leftarrow \text{false}$\;
            }{
                % Line 7: Descriptive text
                determine a shortest path $P$ from $s$ to $t$ in $(G', \gamma')$\;
                
                % Line 8: Multiple assignments
                $\delta \leftarrow \min \{c'(e) : e \in P \}; \delta' \leftarrow \min(\delta, v - \text{val}); \text{val} \leftarrow \text{val} + \delta'$\;
                
                % Line 9
                augment $f$ along $P$ by $\delta'$\;
            }
        }
    }
    \caption{Algoritmo de caminhos de custo mínimo sucessivos \cite{jungnickel}}
    \label{alg:optflow}
\end{algorithm}

 A complexidade do algoritmo é dependente do método utilizado para encontrar o caminho $P$. A implementação padrão utilizaria o algoritmo de Dijkstra, que tem uma complexidade de $O(m + n \log n)$ em uma implementação com filas de prioridade, onde $n$ é o número de vértices e $m$ o número de arestas. Porém, como o grafo residual pode conter arestas com custos negativos, essa implementação não funcionaria corretamente. Para contornar esse problema, temos duas opções principais:

 \begin{itemize}
    \item \textbf{Utilizar Bellman-Ford:} A solução mais simples e direta é utilizar, ao invés do algoritmo de Dijkstra, o algoritmo de Bellman-Ford para encontrar o caminho de custo mínimo $P$. O Bellman-Ford é mais lento, com uma complexidade de $O(nm)$, mas lida corretamente com arestas de custo negativo.
    \item \textbf{Utilizar a técnica de Potenciais:} Técnica que permite que utilizemos o algoritmo de Dijkstra mesmo na presença de arestas de custo negativo. A ideia é manter um \textbf{potencial} para cada vértice $v$ do grafo, chamado de $h(v)$, com valor inicial 0. Definimos o \textbf{custo reduzido} de uma aresta $(u,v)$ como $\gamma'(u,v) = \gamma(u,v) + h(u) - h(v)$. Se os potenciais forem escolhidos de forma adequada, todos os custos reduzidos podem ser garantidos como não-negativos, permitindo o uso do algoritmo de Dijkstra. Além disso, é garantido que os caminhos de custo mínimo no grafo original e no grafo com custos reduzidos são os mesmos. \cite{jungnickel}.
    
    A cada iteração do loop, encontra-se o caminho de custo mínimo $P$ utilizando Dijkstra com os custos reduzidos. Após encontrar o caminho, os potenciais são atualizados para todos os vértices $v$ alcançáveis a partir de $s$ no grafo residual, ajustando-os conforme a distância mínima encontrada pelo Dijkstra, de acordo com a fórmula $h(v) \leftarrow h(v) + d(s,v)$, onde $d(s,v)$ é a distância mínima de $s$ até $v$ no grafo residual com custos reduzidos.

    Abaixo uma adaptação do algoritmo \ref{alg:optflow} utilizando a técnica de potenciais:

    \begin{algorithm}[H]
    \DontPrintSemicolon
    \SetAlgoLined

    \Fn{OPTFLOW\_DIJKSTRA($G, c, s, t, \gamma, v; f, \text{sol}$)}{
        % Inicialização do Fluxo
        \lFor{$e \in E$}{ $f(e) \leftarrow 0$ }
        
        % Inicialização dos Potenciais (assumindo custos iniciais não negativos)
        \lFor{$x \in V$}{ $\pi(x) \leftarrow 0$ }

        $\text{sol} \leftarrow \text{true}, \text{val} \leftarrow 0$\;

        \While{$\text{sol} = \text{true}$ \KwAnd $\text{val} < v$}{
            
            construct the auxiliary network $N' = (G', c', s, t)$\;
            
            % Definição do Custo Reduzido para o Dijkstra
            \For{$e=(u, v) \in E(G')$}{
                define reduced cost: $w(e) \leftarrow \gamma(e) + \pi(u) - \pi(v)$\;
            }

            % Execução do Dijkstra (Lógica omitida, foco no input/output)
            Run Dijkstra on $N'$ using weights $w$ to find shortest path distances $d(\cdot)$ from $s$\;

            \eIf{$t$ is not accessible from $s$ (i.e., $d(t) = \infty$)}{
                $\text{sol} \leftarrow \text{false}$\;
            }{
                % Atualização dos Potenciais (Passo Crucial)
                \lFor{$x \in V$ such that $d(x) < \infty$}{ $\pi(x) \leftarrow \pi(x) + d(x)$ }
                
                Identify shortest path $P$ from $s$ to $t$ based on $d(\cdot)$\;
                
                % O resto segue o algoritmo original, usando as capacidades residuais c'
                $\delta \leftarrow \min \{c'(e) : e \in P \}; \delta' \leftarrow \min(\delta, v - \text{val}); \text{val} \leftarrow \text{val} + \delta'$\;
                
                augment $f$ along $P$ by $\delta'$\;
            }
        }
    }
    \caption{Adaptação do algoritmo \ref{alg:optflow} utilizando Dijkstra com técnica de potenciais}
    \label{alg:optflow_dijkstra}
\end{algorithm}

A complexidade do algoritmo utilizando Dijkstra com a técnica de potenciais é $O(n (m + n \log n))$, onde $n$ é o número de vértices e $m$ o número de arestas do grafo. Isso ocorre porque, em cada iteração do loop principal, executamos o algoritmo de Dijkstra, que tem complexidade $O(m + n \log n)$, e o número máximo de iterações é limitado por $n$, o número de vértices no grafo bipartido.

\end{itemize}



\input{capitulos/4_5_problemas/min-max.tex}

\subsection{Problema do emparelhamento estável (Stable Marriage Problem)}


\newpage
\centering \bibliographystyle{abntex2-alf}
\bibliography{pos_textual/referencias}

\end{document}
